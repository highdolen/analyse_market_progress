% This file was converted to LaTeX by Writer2LaTeX ver. 1.9.9
% see http://writer2latex.sourceforge.net for more info
\documentclass[a4paper]{article}
\usepackage{calc,amsmath,amssymb,amsfonts}
\usepackage[T2A,LGR,T1]{fontenc}
\usepackage[greek,english,russian]{babel}
\usepackage{xcolor,longfbox,fancyhdr}
\usepackage[vmargin=0.7874in,left=1.1811in,right=0.5902in,noheadfoot]{geometry}
\usepackage{enumitem,hyperref}
\hypersetup{colorlinks=true,allcolors=blue,pdfauthor=ФеДор Калин}
\usepackage[pdftex]{graphicx}
\makeatletter\newdimen\@tempdimd\makeatother
% Outline numbering
\setcounter{secnumdepth}{0}
% Text styles
\newcommand\textstyleListLabelcliv[1]{\textbf{#1}}
\newcommand\textstyleListLabelclv[1]{\textbf{#1}}
% Pages
\fancypagestyle{Standard}{\fancyhf{}
  \fancyhead[L]{}
  \fancyfoot[L]{}
  \renewcommand\headrulewidth{0pt}
  \renewcommand\footrulewidth{0pt}
  \renewcommand\thepage{\arabic{page}}
}
\pagestyle{Standard}
\author{ФеДор Калин}
\date{2025-12-10}
\begin{document}
\begin{enumerate}[series=listWWNumxviii,label=\textstyleListLabelcliv{\arabic*.},ref=\arabic*]
\item
\begin{enumerate}[series=listWWNumxviii,label=\textstyleListLabelclv{\arabic{enumi}.\arabic*.},ref=\arabic{enumi}.\arabic*]
\item 
\bigskip
\end{enumerate}
\end{enumerate}
{\centering
ФИО разработчиков: Калин Федор Валерьевич, Карастоянов Кирилл Сергеевич.
\par}

{\centering
\textbf{1.2.}
\par}

{\centering
Группа: Д-Э342
\par}

{\centering
\textbf{1.3. Тема проекта}
\par}

{\centering
{\textquotedbl}Анализ тренировочного прогресса на Python{\textquotedbl}
\par}

{\centering
\textbf{1.4. Цель проекта}
\par}

Исследовать и продемонстрировать возможности и ограничения инструментов и методов языка программирования Python для
анализа и моделирования данных тренировочного прогресса.

{\centering
\textbf{1.5. Задачи проекта}
\par}

Выполнить загрузку и предобработку исходных данных: очистка пропусков и нулей, коррекция типов, возможное устранение
аномалий. 

Провести первичный исследовательский анализ: визуализировать временные ряды ключевых метрик, их распределения и тренды;
построить корреляционную матрицу для оценки взаимосвязей между показателями. 

Разработать и оценить модели регрессии (линейную и множественную), чтобы проверить, насколько хорошо можно предсказать
целевую числовую метрику на основе остальных доступных признаков и оценить силу линейных зависимостей. 

Построить модели классификации (например, дерево решений, KNN), чтобы попытаться предсказать категориальное изменение
целевого показателя (например, рост или снижение) и оценить их качество. 

Провести кластерный анализ (K-Means), чтобы сгруппировать наблюдения по схожим характеристикам и попытаться выявить
типичные паттерны или состояния системы. 

Использовать методы нейронных сетей для регрессии (MLP) $\text{\textgreek{—}}$ применить более гибкий, нелинейный подход
к прогнозированию целевой переменной, чтобы оценить, насколько он даёт результат лучше простых линейных моделей.


\bigskip

{\centering
\textbf{1.6. Аннотация}
\par}

Проект посвящён анализу личных данных тренировочного прогресса за фиксированный период. Целью работы является
демонстрация возможностей языка программирования Python и его инструментария для комплексного исследования и анализа
данных, связанных с фитнесом и здоровьем. В рамках проекта выполнен полный цикл анализа: от предобработки исходных
данных (включая удаление пропусков и аномалий, корректировку типов данных и фильтрацию нерелевантных значений) до
применения современных методов машинного обучения.


\bigskip

Проведён детальный исследовательский анализ с визуализацией временных рядов, изучением распределений и трендов ключевых
метрик (вес, потребление калорий, активность), а также оценкой взаимосвязей показателей с помощью корреляционного
анализа. Для решения задач прогнозирования и классификации были реализованы и сравнены модели регрессии (линейная и
множественная) для предсказания целевых числовых показателей, методы классификации (дерево решений и метод k-ближайших
соседей) для категоризации типов тренировочных дней, алгоритм кластеризации K-Means для выявления паттернов активности
и нейросетевая модель MLPRegressor для нелинейного прогнозирования. Особое внимание уделено сравнительной оценке
эффективности и устойчивости построенных моделей.


\bigskip

Результаты работы показывают, какие подходы позволяют выявлять значимые закономерности в персональных фитнес-данных,
прогнозировать ключевые изменения и сегментировать типы активностей, а какие демонстрируют ограниченную применимость.
Проект служит наглядным примером того, как с помощью Python и библиотек для анализа данных и машинного обучения можно
системно исследовать динамику тренировочного прогресса $\text{\textgreek{—}}$ от первичной обработки до построения
интерпретируемых моделей и формирования практических выводов для оптимизации режима.

{\centering
\textbf{2. Предобработка данных.}
\par}

Первый этап проекта - предобработка данных, которая является фундаментальным для обеспечения качества всего последующего
анализа. Целью этого этапа была проверка целостности данных, корректировка форматов и устранение потенциальных
артефактов, которые могли бы исказить результаты моделирования.


\bigskip

Первым шагом была проведена первичная диагностика структуры датасета. С помощью метода
{\textasciigrave}.info(){\textasciigrave} была получена общая картина о типах данных в каждом столбце и количестве
заполненных записей, что позволило подтвердить соответствие ожидаемым форматам. Для числовых признаков было
сгенерировано статистическое описание методом {\textasciigrave}.describe(){\textasciigrave}, которое показало ключевые
параметры распределений, включая средние значения, разброс и экстремальные точки. Далее был выполнен анализ на наличие
пропущенных значений. Проверка методом {\textasciigrave}.isnull().sum(){\textasciigrave} выявила отсутствующие записи в
числовых колонках. Для сохранения целостности временного ряда и объема наблюдений было принято решение заменить эти
пропуски средними арифметическими значениями по соответствующим признакам. Затем из набора данных были удалены все
обнаруженные полные дубликаты строк, чтобы исключить риск искажения статистик и моделей из-за повторяющихся записей.
После завершения всех процедур очистки был выведен итоговый размер датасета, что подтвердило сохранение его
репрезентативности.


\bigskip

В результате проведения описанных процедур был сформирован очищенный, структурированный и согласованный набор данных,
который стал надежной основой для проведения углубленного исследовательского анализа и построения прогнозных моделей
машинного обучения.


\lfbox[margin=0mm,border-style=none,padding=0mm,vertical-align=top]{\includegraphics[width=6.4965in,height=4.3681in]{a0000-img001.png}}


{\centering
\textbf{3. Визуализация данных.}
\par}


\bigskip

После завершения предобработки следующим ключевым этапом стала визуализация данных, цель которой $\text{\textgreek{—}}$
получить интуитивное понимание структуры, распределения и потенциальных аномалий в числовых признаках. Графическое
представление информации позволяет выявить скрытые закономерности, которые могут быть неочевидны при анализе только
табличных сводок.

Особое внимание было уделено анализу распределения и разброса значений по каждому из числовых признаков. Для этой задачи
использовался инструмент {\textasciigrave}boxplot{\textasciigrave} (ящик с усами). Данная визуализация наглядно
отображает медиану, межквартильный размах, а также потенциальные выбросы $\text{\textgreek{—}}$ наблюдения, которые
существенно отклоняются от общего распределения. Анализ бокс-плота позволил качественно оценить вариативность данных по
каждому измеряемому показателю, идентифицировать признаки с наибольшим разбросом и отметить те метрики, где
присутствуют экстремальные значения, требующие дополнительного внимания при построении моделей. Этот этап заложил
основу для более детального исследования аномалий и понимания масштабов данных перед переходом к корреляционному
анализу и моделированию.

\centering
\lfbox[margin=0mm,border-style=none,padding=0mm,vertical-align=top]{\includegraphics[width=4.8811in,height=3.2728in]{a0000-img002.png}}
\par

\bigskip


\bigskip


\bigskip

{\centering
\textbf{\textcolor[HTML]{1D2125}{3.1. Корреляционный анализ.}}
\par}

Корреляционный анализ\textbf{ }показал практически полное отсутствие выраженных линейных взаимосвязей между признаками.
Все коэффициенты корреляции оказались расположены в диапазоне от –0.10 до +0.14, что говорит об их чрезвычайно слабой
связи. Масса тела практически не коррелирует с остальными характеристиками, то есть в рамках исследуемого периода она
остаётся независимой от калорийности рациона, потребления белка, длительности тренировок и количества шагов.
Потребление калорий демонстрирует слабую положительную связь с количеством белка и едва заметную отрицательную
зависимость со степенью активности, выраженной количеством шагов. Потребление белка также почти не связано с остальными
признаками, лишь минимально коррелируя с калорийностью и длительностью тренировок. Продолжительность тренировок
демонстрирует слабую положительную связь с количеством шагов, что вполне логично, но эта связь крайне слаба. В целом
анализ показал, что признаки ведут себя практически независимо друг от друга, и это предопределяет низкую эффективность
линейных моделей.


\bigskip

\centering
\lfbox[margin=0mm,border-style=none,padding=0mm,vertical-align=top]{\includegraphics[width=5.8193in,height=3.8264in]{a0000-img003.png}}
\par
{\centering
\textbf{\textcolor[HTML]{1D2125}{3.2. Регрессионный анализ (линейная регрессия, множественная линейная регрессия).}}
\par}

{\centering
\textbf{3.2.1 Простая линейная регрессия}
\par}

Линейный регрессионный анализ\textbf{, }направленный на выявление зависимости между количеством шагов и длительностью
тренировок, подтвердил низкую степень линейной зависимости между этими признаками. Значение коэффициента регрессии
указывает на увеличение количества шагов примерно на 17 единиц при увеличении тренировки на одну минуту, что слишком
мало, чтобы говорить о значимом влиянии. Свободный член модели отражает среднее количество шагов в дни с нулевой
длительностью тренировок. Коэффициент детерминации оказался крайне низким и составил всего 0.019, что означает
способность модели объяснить лишь 1.9\% вариации целевой переменной. Это говорит о том, что шаговая активность
определяется преимущественно повседневной и бытовой активностью, а не тренировками.


\lfbox[margin=0mm,border-style=none,padding=0mm,vertical-align=top]{\includegraphics[width=4.6465in,height=0.9165in]{a0000-img004.png}}



\bigskip

{\centering
\textbf{\textcolor[HTML]{1D2125}{3.2.2 Множественная линейная регрессия}}
\par}

Множественная линейная регрессия,\textbf{ }в которой одновременно учитывались длительность тренировки, калорийность
рациона и количество потребляемого белка, также продемонстрировала крайне слабую зависимость предсказываемой переменной
от набора признаков. Коэффициент детерминации равен 0.028, то есть модель объясняет менее 3\% вариации количества
шагов. Это говорит о практически полном отсутствии линейной связи между входными признаками и целевой переменной.
Коэффициенты модели позволяют лишь обозначить направления влияния, но сами по себе не несут значимой предсказательной
ценности. Вклад длительности тренировок оказался положительным, потребление калорий $\text{\textgreek{—}}$ слабым
отрицательным, а потребление белка $\text{\textgreek{—}}$ минимально положительным. Тем не менее столь низкий R²
показывает, что модель фактически не способна выполнять качественное предсказание шагов. Следовательно, множественная
линейная регрессия оказывается слабым инструментом для прогнозирования активности в данном наборе данных.

\centering
\lfbox[margin=0mm,border-style=none,padding=0mm,vertical-align=top]{\includegraphics[width=6.4902in,height=0.9791in]{a0000-img005.png}}
\par
{\centering
\textbf{\textcolor[HTML]{1D2125}{4. Классификация данных. Деревья решений.}}
\par}

Анализ классификации с использованием деревьев\textbf{ }решений\textbf{ }показал, что самыми значимыми признаками для
разделения пользователей по уровню активности являются потребление калорий, длительность тренировок, потребление белка
и вес. Корневым фактором модели стало количество потребляемых калорий, что свидетельствует о его важной роли при
определении уровня активности. При низком уровне калорий ключевое влияние начинают оказывать длительность тренировок и
потребление белка, а при высоком уровне калорий существенным становится вес человека. Таким образом, модель дерева
решений выявляет, что активность определяется сразу несколькими факторами, причём их значимость меняется в зависимости
от начальных условий. Несмотря на это, структура дерева показывает, что между классами присутствует пересечение, и
данные не всегда позволяют идеально разделить наблюдения.

\centering
\lfbox[margin=0mm,border-style=none,padding=0mm,vertical-align=top]{\includegraphics[width=6.4965in,height=3.8244in]{a0000-img006.png}}
\par
\centering
\lfbox[margin=0mm,border-style=none,padding=0mm,vertical-align=top]{\includegraphics[width=6.4965in,height=3.2835in]{a0000-img007.png}}
\par

\bigskip

{\centering
\textbf{\textcolor[HTML]{1D2125}{5. Классификация данных. Алгоритм KNN.}}
\par}

Классификация с использованием алгоритма\textbf{ }KNN\textbf{ }позволила разделить пользователей на активных и
неактивных после предварительной балансировки классов и подбора оптимального числа соседей. Модель показала умеренную
точность около 57 процентов. Она выявила, что признаки, связанные с питанием и тренировками, позволяют приблизительно
разделять пользователей по уровню активности, однако значительного перекрытия классов избежать не удалось. Это
означает, что данные содержат некоторое количество информации об уровне активности, но для более надёжного разделения
потребовались бы дополнительные признаки или более сложные методы классификации.

\centering
\lfbox[margin=0mm,border-style=none,padding=0mm,vertical-align=top]{\includegraphics[width=4.9925in,height=2.9327in]{a0000-img008.png}}
\par
\centering
\lfbox[margin=0mm,border-style=none,padding=0mm,vertical-align=top]{\includegraphics[width=6.4965in,height=5.0311in]{a0000-img009.png}}
\par

\bigskip

{\centering
\textbf{\textcolor[HTML]{1D2125}{6. Кластерный анализ. Алгоритм K-Means.}}
\par}

Кластеризация методом K-Means\textbf{ }позволила выделить три отчётливо различающихся группы пользователей. Первый
кластер характеризуется умеренным уровнем активности, средними значениями по всем признакам и относительно невысоким
количеством шагов. Второй кластер представляет собой группу людей, придерживающихся высококалорийного и богатого белком
рациона, с относительно длительными тренировками, но небольшим количеством шагов, что может соответствовать профилю
пользователей, ориентированных на силовые тренировки. Третий кластер объединяет наиболее активных пользователей,
которые тренируются дольше всех, много ходят, но потребляют меньше калорий по сравнению с другими группами. Таким
образом, кластеризация показала, что пользователи формируют естественные группы с разными типами поведения в питании и
активности.


\bigskip

\centering
\lfbox[margin=0mm,border-style=none,padding=0mm,vertical-align=top]{\includegraphics[width=6.4965in,height=4.289in]{a0000-img010.png}}
\par
\centering
\lfbox[margin=0mm,border-style=none,padding=0mm,vertical-align=top]{\includegraphics[width=6.4965in,height=5.0736in]{a0000-img011.png}}
\par
\centering
\lfbox[margin=0mm,border-style=none,padding=0mm,vertical-align=top]{\includegraphics[width=6.4965in,height=1.4929in]{a0000-img012.png}}
\par
{\centering
\textbf{\textcolor[HTML]{1D2125}{7. Нейронные сети.}}
\par}

Нейронная сеть, использованная для предсказания количества шагов на основе признаков питания и тренировок, показала
крайне низкое качество прогнозирования. Большое значение MSE и отрицательный коэффициент детерминации R²
свидетельству044Eт о том, что модель не способна улавливать закономерности в данных. Это обусловлено слабой связью
признаков с целевой переменной, а также малым объёмом выборки. В таких условиях нейронная сеть демонстрирует ожидаемо
низкую эффективность, что делает её непригодной для качественного предсказания в данном случае.


\lfbox[margin=0mm,border-style=none,padding=0mm,vertical-align=top]{\includegraphics[width=6.4071in,height=0.7189in]{a0000-img013.png}}

\end{document}
